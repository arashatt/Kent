
% https://opensource.com/article/22/8/pdf-latex
\documentclass[10pt,a4paper]{article}
\usepackage[T1]{fontenc}
\usepackage{minted}
\usepackage{todonotes} % for the todo notes (optional)
\newcommand{\definition}[2]{\textbf{#1}: #2}

\begin{document}
	\section*{mio}
	mio is a Rust crate for non-blocking I/O API interface for building \textbf{high performance I/O apps}.
	\begin{enumerate}

		\item Creating a Poll. reads events from OS and puts them in Event
		\item Register an Event source
		\item Create an Event loop

	\end{enumerate}
	At the end you’ll have a very small (but quick) TCP server that accepts connections and then drops (disconnects) them.
	

\inputminted{rust}{fig01.rs}

\subsection*{Poll}
Poll is a struct allows a program to monitor a large number of \texttt{event::Source}s. It waits until one of them is ready for some class of operations e.g read or write.\footnote{\texttt{event }\texttt{token} and \texttt{interest} are to be interpolated}

\inputminted{rust}{fig02.rs}

\subsection*{event::Source}

\subsection*{Token}
token is a wrapper around \texttt{usize} and is used as an argument to \texttt{Registry::register } and \texttt{Registry::reregister}.
\section*{epoll}
The implementation of epoll uses epoll systemcall defined at \texttt{sys/epoll.h}.
According to the official documentation, it is monitoring multiple file descriptions to see if \texttt{I/O} operation is possible on any of them.

\begin{itemize}
	\item interest list (epoll set)
	\item ready list (a set of references to the interest list)
\end{itemize}
Other systemcalls related to epoll are the following:
\begin{itemize}
	\item epoll\_create
	\item epoll\_create1
	\item epoll\_cntl
	\item epoll\_wait
\end{itemize}
\definition{epoll\_create1}{if the argument is 0, it has the same functionality of epoll\_create. the other argument we can pass to this syscall is \textbf{FD\_CLOEXEC}}.\\
\definition{epoll\_wait}{blocks the current thread if no event is already available.}

\section*{A code example}
\inputminted{rust}{fig03.rs}
\end{document}
